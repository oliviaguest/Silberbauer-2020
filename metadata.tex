% DO NOT EDIT - automatically generated from metadata.yaml

\def \codeURL{}
\def \codeDOI{}
\def \codeSWH{swh:1:rev:95a00613e591becafb911c97d9995a8cca68002d}
\def \dataURL{http://www.riv.co.za/wv/data/CHEMDATA.DAT}
\def \dataDOI{}
\def \editorNAME{Olivia Guest}
\def \editorORCID{0000-0002-1891-0972}
\def \reviewerINAME{Nicolas P. Rougier}
\def \reviewerIORCID{0000-0002-6972-589X}
\def \reviewerIINAME{}
\def \reviewerIIORCID{}
\def \dateRECEIVED{31 March 2020}
\def \dateACCEPTED{27 July 2020}
\def \datePUBLISHED{20 August 2020}
\def \articleTITLE{Re ReScience challenge: Geographical Trends in the Water Chemistry of Wetlands in the South-Western Cape Province, South Africas}
\def \articleTYPE{Reproduction}
\def \articleDOMAIN{Aquatic Science}
\def \articleBIBLIOGRAPHY{Silberbauer-2020.bib}
\def \articleYEAR{2020}
\def \reviewURL{https://github.com/ReScience/submissions/issues/25}
\def \articleABSTRACT{A map of water quality with Maucha ionic diagrams at monitoring sites provides a simple visual comparison of the major ions in water samples. The ratios of the major ions provide an overall impression of the suitability of water for biota. Clusters of similarly shaped symbols highlight geographical distributions that may warrant further investigation. When first developed in the 1920s, the symbols were calculated and drawn by hand, a time-consuming process that limited their usefulness. Construction of the symbols is simplified using a computer program, and I describe an early procedure written in Turbo Pascal. The automated production of symbols allows for enhancements such as colour-coding of the ions and scaling in proportion to the total ion concentration. This Pascal program was the basis for further developments during the past thirty years, in geographical information systems and in R.}
\def \replicationCITE{Silberbauer, M.J. and King, J.M., 1991. Geographical Trends in the Water Chemistry of Wetlands in the South-Western Cape Province, South Africa. Southern African Journal of Aquatic Sciences, 17 (1/2), 82�88.}
\def \replicationBIB{}
\def \replicationURL{http://dx.doi.org/10.1080/10183469.1991.9631315}
\def \replicationDOI{10.1080/10183469.1991.9631315}
\def \contactNAME{Michael Silberbauer}
\def \contactEMAIL{Michael.Silberbauer@gmail.com}
\def \articleKEYWORDS{water chemistry, maucha, wetland, data visualisation, pascal}
\def \journalNAME{ReScience C}
\def \journalVOLUME{6}
\def \journalISSUE{1}
\def \articleNUMBER{13}
\def \articleDOI{10.5281/zenodo.3961862}
\def \authorsFULL{Michael Silberbauer}
\def \authorsABBRV{M. Silberbauer}
\def \authorsSHORT{Silberbauer}
\title{\articleTITLE}
\date{}
\author[1,\orcid{0000-0001-9822-4149}]{Michael Silberbauer}
\affil[1]{Retired since 2020-01-31: formerly Resource Quality Information Services, Pretoria, South Africa}
